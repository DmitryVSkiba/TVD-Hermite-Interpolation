%% LyX 2.1.0 created this file.  For more info, see http://www.lyx.org/.
%% Do not edit unless you really know what you are doing.
\documentclass[english]{article}
\usepackage[T1]{fontenc}
\usepackage[utf8]{luainputenc}
\usepackage{babel}
\begin{document}
Алгоритм расчета полинмов Эрмита с обеспечением монотонности согласно
алгоритму изложенному в работе Accurate Monotonicity Preserving Cubic
Interpolation, J. M. Hyman, Society of Industrial and Applied Mathematics
Journal of Scientific and Statistical Computing, 4, No. 4, (December
1983) pp. 645-654.( метод Fritsch-Butland) 

1. Исходные данные для алгоритма Одномерный массив точек в котрорых
интерполируется функция (в данном случае Ti) длиной N штук, нумерация
начинается с 0, индекс последнего элемента массива N-1 Одномерный
массив значений функции в точках в данном случае массив значений открытия
крана в \% (Yi) длиной N штук, нумерация начинается с 0, индекс последнего
элемента массива N-1 

2. Подготовительные операции Рассчитываются следующие массивы h\_i,∆\_i
длины N-1, нумерация начинается с 0, последний элемент массива имеет
индекс N-2 расчитываются дополнительные массивы значений для шагов
по независимой переменной длины N-2, нумерация начинается с 0, последний
элемент массива имеет индекс N-3: ,

3. Процедура ограничения производных для обеспечения монотонности.
По данным подготовительной операции находятся массивы  и dmax , нумерация
начинается с 0, длины N-2, последний элемент массива имеет индекс
N-3 по следующим формулам: По данным этих массивов строится массив
длины N-2 по формуле 
\end{document}
